\documentclass[a4paper]{article}
\usepackage[margin=2cm]{geometry}

\usepackage{fontspec}			% utf-8 suport
\usepackage{amsmath}			% Math utilities
\usepackage{amssymb}			% Math symbols
\usepackage{amsfonts}			% More math fonts
\usepackage{mathtools}			% More math tools (rcases)
\usepackage[makeroom]{cancel}	% Math cancel
\usepackage[catalan]{babel} 	% Language 
\usepackage{enumitem}			% To resume the enumeration
\usepackage{float}				% Better floating options

\setlength{\parindent}{0pt}
\setlength{\parskip}{0.2cm}

\title{\textsc{APA Problemes} \\ Problema 5 Propietats elàstiques d'una molla}
\author{Joan Marcè i Igual}
\date{}


\begin{document}
\maketitle

\textbf{Volem determinar les propietats elàstiques d'una molla usant diferents pesos i mesurant la deformació que es produeix. La llei de Hooke relaciona la longitud $l$ i la força $F$ que exerceix el pes com:}
$$
e + kF = l
$$
\textbf{on $e,k$ són constants de la llei, que es volen determinar. S'ha realitzat un experiment i obtingut les dades:}
\begin{table}[H]
	\centering
	\begin{tabular}{l|rrrrr}
		$F$ & 1 & 2 & 3 & 4 & 5 \\
		\hline
		$l$ & 7.97 & 10.2 & 14.2 & 16.0 & 21.2
	\end{tabular}
\end{table}

\begin{enumerate}
	\item \textbf{Plantegeu el problema com un problema de mínims quadrats}
\end{enumerate}
S'han d'obtenir les funcions de base:
$$
f(\vec{w}, \vec{x}) = \sum_{i=0}^{M-1} w_i \phi_i(\vec{x}) \implies l(F) =
\underbrace{e}_{w_0}·\underbrace{1}_{\phi_0} +
\underbrace{k}_{w_1}·\underbrace{F}_{\phi_1} 
$$
Així doncs s'obtenen els següents paràmetres:
$$
\vec{w} = 
\begin{pmatrix}
e \\
k
\end{pmatrix}
\qquad
\vec{\phi}(\vec{x}) = \vec{\phi}(F) =
\begin{pmatrix}
1 \\
F
\end{pmatrix}
$$
I per tant les matrius de disseny i resultats són les següents:
$$
\Phi = 
\begin{pmatrix}
1 & 1 \\
1 & 2 \\
1 & 3 \\
1 & 4 \\
1 & 5
\end{pmatrix}
\qquad
t = 
\begin{pmatrix}
7.97 \\
10.2 \\
14.2 \\
16.0 \\
21.2
\end{pmatrix}
$$
I el sistema resultat és el següent:
$$
(\Phi^T\Phi)w = \Phi^T t \implies w = (\Phi^T\Phi)^{-1} \Phi^T t 
$$

\begin{enumerate}[resume]
	\item \textbf{Resoleu-lo amb el mètode de la matriu pseudo-inversa}
\end{enumerate}
La matriu pseudo-inversa ($\Phi^+$) es calcula de la següent manera:
$$
\Phi^+ = (\Phi^T\Phi)^{-1}\Phi^T
$$
Així doncs mitjançant \texttt{R} s'obtenen els següents valors per a la matriu pseudo-inversa:
$$
\Phi^+ =
\begin{pmatrix}
0.8 & 0.5 & 0.2 & -0.1 & -0.4\\
-0.2 & -0.1 & 0.0 & 0.1 & 0.2\\
\end{pmatrix}
\implies 
w = 
\begin{pmatrix}

\end{pmatrix}
$$

\begin{enumerate}[resume]
	\item \textbf{Resoleu-lo amb el mètode basat en SVD}
\end{enumerate}

Es busquen els valors singulars de la matriu de disseny $\Phi$.
$$
U = 
\begin{pmatrix}
0.16 & 0.76 \\
0.29 & 0.47 \\
0.41 & 0.18 \\
0.54 & -0.11 \\
0.66 & -0.40 \\
\end{pmatrix}
\qquad
D =
\begin{pmatrix}
7.69 & 0 \\
0 & 0.92 \\
\end{pmatrix}
\qquad
V = 
\begin{pmatrix}
0.27 & 0.96 \\
0.96 & -0.27 \\
\end{pmatrix}
$$
El valor de $\vec{w}$ es pot trobar mitjançant:
$$
\vec{w} = V·diag\left(\frac{1}{\lambda_i}\right)·U^T·t
$$
$$
\vec{w} = 
$$




\end{document}