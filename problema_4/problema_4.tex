\documentclass[a4paper]{article}
\usepackage[margin=2cm]{geometry}

\usepackage{fontspec}			% utf-8 support
\usepackage{amsmath}			% Math utilities
\usepackage{amsfonts}			% Math fonts
\usepackage[makeroom]{cancel}	% Math cancel (crossed symbols)
\usepackage{enumitem}			% To resume enumerations

\setlength{\parindent}{0pt}
\setlength{\parskip}{0.2cm}

\title{\textsc{APA Problemes} \\ Problema 5 La fàbrica de píndoles I}
\author{Joan Marcè i Igual}
\date{}

\begin{document}
\maketitle

\textbf{La companyia farmacèutica \emph{Nice Pills} ha construït una cinta transportadora que porta dues \emph{classes} de píndoles (adequades per dos tipus de malalties diferents), que anomenem $C_1$ i $C_2$. Aquestes píndoles surten en dos colors: $\{yellow, white\}$, que són detectats per una càmera. La companyia fabrica píndoles en proporcions $P(C_1) = \frac{1}{3}$, $P(C_2) = \frac{2}{3}$. Se'ns facilita també informació sobre la distribució del color per cada classe: $P(yellow|C_1) = \frac{1}{5}$, $P(white|C_1) = \frac{4}{5}$ i $P(yellow|C_2)=\frac{2}{3}$, $P(white|C_2)=\frac{1}{3}$. Es demana:}


\begin{enumerate}
	\item \textbf{Quina és la probabilitat d'error si no s'utilitza el color per classificar?}
\end{enumerate}
Si no s'utilitza el color per classificar es defineix la regla següent:
$$
R_1: \text{ la classe de } X = 
\begin{cases}
C_1 \text{ si } P(C_1) > P(C_2) \\
C_2 \text{ si } P(C_1) < P(C_2) \\
\end{cases}
$$
Per tant la probabilitat d'error serà la d'escollir un element com una classe però que ho sigui d'una altra per tant:
$$
P_{error}(R_1) = \min(P(C_1),P(C_2)) = P(C_1) = \boxed{\frac{1}{3}}
$$

\begin{enumerate}[resume]
	\item \textbf{Calcular les probabilitats $P(yellow)$ i $P(white)$ i les probabilitats $P(C_1|yellow)$, $P(C_2|yellow)$, $P(C_1|white)$, $P(C_1|yellow)$.}
\end{enumerate}
\begin{align*}
& P(yellow) = P(yellow|C_1)P(C_1) + P(yellow|C_2)P(C_2) = 
\frac{1}{5}·\frac{1}{3} + \frac{2}{3}·\frac{2}{3} = 
\frac{1}{15} + \frac{4}{9} = \boxed{\frac{23}{45}} \\
& P(white) = 1 - P(yellow) = 1 - \frac{23}{45} = 
\boxed{\frac{22}{45}} \\
& P(C_1|yellow) = \frac{P(yellow|C_1)P(C_1)}{P(yellow)} = 
\frac{\frac{1}{5}·\frac{1}{3}}{\frac{23}{45}} = \boxed{\frac{3}{23}} \\
& P(C_2|yellow) = 1 - P(C_1|yellow) = 1 - \frac{3}{23} = 
\boxed{\frac{20}{23}} \\
& P(C_1|white) = \frac{P(white|C_1)P(C_1)}{P(white)} =
\frac{\frac{4}{5}·\frac{1}{3}}{\frac{22}{45}} =
\boxed{\frac{6}{11}} \\
& P(C_2|white) = 1 - P(C_1|white) = 1 - \frac{6}{11} =
\boxed{\frac{5}{11}}
\end{align*}


\begin{enumerate}[resume]
	\item \textbf{Quina és la decisió òptima per pastilles \emph{yellow}? I per pastilles \emph{white}? Quins són els \emph{odds} en ambdós casos?}
\end{enumerate}
\begin{align*}
R_{yellow}: \text{ la classe de } X = &
\begin{cases}
C_1 \text{ si } P(C_1|yellow) > P(C_2|yellow) \\
C_2 \text{ si } P(C_1|yellow) < P(C_2|yellow) \\
\end{cases} \\
& \begin{cases}
P(C_1|yellow) = \frac{3}{23} \\
P(C_2|yellow) = \frac{20}{23}
\end{cases}
\implies X = C_2
\end{align*}
Per tant la decisió òptima per pastilles \emph{yellow} és la classe $C_2$.

\begin{align*}
R_{white}: \text{ la classe de } X = &
\begin{cases}
C_1 \text{ si } P(C_1|white) > P(C_2|white) \\
C_2 \text{ si } P(C_1|white) < P(C_2|white) \\
\end{cases} \\
& \begin{cases}
P(C_1|white) = \frac{6}{11} \\
P(C_2|white) = \frac{5}{11}
\end{cases}
\implies X = C_1
\end{align*}
Per tant la decisió òptima per pastilles \emph{white} és la classe $C_1$.

\begin{enumerate}[resume]
	\item \textbf{Quina és la probabilitat d'error si s'utilitza el color per classificar? Perquè és millor que la de l'apartat 1?}
\end{enumerate}
Si s'usa el color per classificar es defineix la següent regla, que és combinació de les dues regles definides anteriorment:
$$
R_{color}: \text{ la classe de } X = 
\begin{cases}
C_1 \text{ si } color = white \\
C_2 \text{ si } color = yellow \\
\end{cases}
$$

La probabilitat d'error és la de dir que un element pertany a una classe i que en realitat pertanyi a l'altra classe per tant:
\begin{align*}
P_{error}(R_{color}) & = 
\min(P(C_1|yellow),P(C_2|yellow))P(yellow) + 
\min(P(C_1|white),P(C_2|white))P(white) = \\
& = P(C_1|yellow)P(yellow) + P(C_2|white)P(white) = 
\frac{3}{23}·\frac{23}{45} + \frac{6}{11}·\frac{22}{45} = \\
& = \frac{3}{45} + \frac{12}{45} = \boxed{\frac{1}{3}}
\end{align*}




\end{document}
